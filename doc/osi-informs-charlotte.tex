\documentclass{beamer}
\usepackage{amsmath,amssymb,listings}
\usepackage{palatino}

%\usetheme{Berlin}

\title{The COIN-OR \\ Open Solver Interface 2.0}
\author{Lou Hafer\inst{1} \and Matthew Saltzman\inst{2}}
\institute{
  \inst{1}
  Department of Computer Science \\
  Simon Fraser University
  \and
  \inst{2}
  Department of Mathematical Sciences \\
  Clemson University.
}
\date{INFORMS Annual Meeting \\ Charlotte, North Carolina \\ November 15, 2011}

\begin{document}

\lstset{language=C++}

\begin{frame}
  \titlepage
\end{frame}

\AtBeginSection[]
{
 \begin{frame}
  \frametitle{Outline}
  \small
  \tableofcontents[currentsection,hideothersubsections]
  \normalsize
 \end{frame}
}

\section{Introduction}
\begin{frame}
  \frametitle{What is OSI?}

  \begin{itemize}
  \item A cross-solver API
  \item Lower level than most solver APIs
    \begin{itemize}
    \item Intended as a ``crossbar switch'' to connect applications to
      solvers
    \item Instance management
    \item Algorithm control (e.g., pivot-level simplex) is a goal
      % (honored more often than not in the breach)
    \end{itemize}
  \item One of the original COIN-OR projects (a product of impetuous
    youth and inexperience)
  \end{itemize}
\end{frame}

\begin{frame}
  \frametitle{Design Objectives for OSI 2.0}
  \begin{itemize}
  \item Transparent plugin framework for dynamic loading of back
    ends and solver libraries
  \item Good programming practice---clean separation of interface
    and implementation, based on standard design patterns, etc.
  \item Small, focused, reusable, extensible APIs
  \item Support for interaction of APIs
  \item Ease of use, ease of shim generation
  \end{itemize}
\end{frame}

\section{The Plugin API}

\begin{frame}
  \frametitle{Dynamic Loading of Solver Engine}
  \begin{itemize}
  \item Solver Engine loaded at runtime, not needed at link time
  \item \lstinline|dlopen()|, \lstinline|dlsym()|, etc., in Linux, other
    calls in Windows and other Unix systems
  \item Cross-platform libraries for this task (GNOME glib, GNU
    libtool)
  \item Mechanics should be hidden from users
  \end{itemize}
\end{frame}

\begin{frame}
  \frametitle{Plugin Management Architecture: The Manager}
  \begin{itemize}
  \item The \lstinline|Osi2PluginManager| class supports
    the loading and unloading of external libraries.
    \begin{itemize}
    \item Get/set default path
    \item Load/unload a single plugin library
    \item Load all plugin libraries in a directory
    \item Unload all plugin libraries
    \item Provide services to plugin
    \end{itemize}
  \item Plugin loading is atomic
  \item Multiple libraries can provide the same API
  \end{itemize}
\end{frame}

\begin{frame}
  \frametitle{Plugin Management Architecture: The Plugin}

  \begin{itemize}
  \item Plugin pulls in solver back-end libraries and provides a
    factory for API objects
  \item A plugin library needs an \lstinline|initPlugin()| function
    with `C' linkage.
    \begin{itemize}
    \item Plugin manager sends pointers to functions for API
      registration and services, including plugin state management
    \item Supports distinct states for plugin library and each API
    \item \lstinline|initPlugin()| registers API create and destroy
      functions, returns state management object and cleanup
      function
    \end{itemize}
  \item Can provide shims written in C++ or C (wrapped in C++
    adapters).
  \end{itemize}
\end{frame}

\begin{frame}
  \frametitle{Loading Solver Back End}

  \begin{itemize}
  \item Heavyweight shim
    \begin{itemize}
    \item Link with solver library so all functions are loaded
      automatically at startup
    \item Dynamic solver library?
    \end{itemize}
  \item Lightweight shim
    \begin{itemize}
    \item Dynamically load solver library
    \item Load functions from solver library on first use as needed
      by each shim call
    \item Requires `C' linkage
    \end{itemize}
  \end{itemize}
\end{frame}

\section{The Control API}

\begin{frame}
  \frametitle{The Control API---A User-Friendly PM Interface}

  \begin{itemize}
  \item Loads and unloads plugin libraries
  \item Creates and destroys objects
  \item User can specify library that creates an object (or not)
  \item Control API maintains object identification information
  \item Utility functions
  \end{itemize}
\end{frame}

\begin{frame}[fragile]
  \frametitle{An Example: Create Several APIs}

  The main loop:
  \small
\begin{lstlisting}
std::vector<std::string> solvers;
solvers.push_back("clp");
solvers.push_back("clpHeavy");
solvers.push_back("glpkHeavy");
std::vector<std::string>::const_iterator iter;
for (iter = solvers.begin(); iter != solvers.end(); 
     iter++) {
  std::string solverName = iter;
  retval = testControlAPI(solverName,dfltSampleDir);
  totalErrs += retval;
}
\end{lstlisting}
\end{frame}

\begin{frame}[fragile]
  \frametitle{An Example: Create Several APIs (cont.)}

  The \lstinline|testControlAPI()| function:
  \small
\begin{lstlisting}
API apiObj = nullptr;
if (ctrlAPI.createObject(apiObj,"Osi1"))
  errcnt++;
else {
  Osi1API *osi = dynamic_cast<Osi1API *>(apiObj);
  std::string exmip1Path = dfltSampleDir+"/brandy.mps";
  osi->readMps(exmip1Path.c_str());
  Osi1API *o2 = osi->clone();
  if (ctrlAPI.destroyObject(apiObj)) errcnt++;
  o2->initialSolve();
  if (!o2->isProvenOptimal()) errcnt++ ;
  if (ctrlAPI.destroyObject(o2)) errcnt++;
}
\end{lstlisting}
\end{frame}

\section{The Feature API: Models and Solvers}

\begin{frame}
  \frametitle{Design Concepts}

  \begin{itemize}
    \item Object interactions managed using inheritance or
      composition
    \item Composition:
      \begin{itemize}
      \item an object of one type contains a reference to an object
        of another type
      \item The owner object calls the owned objects methods
      \item The owner can pass its \lstinline|this| pointer to the owned
        object's methods if the owned object must act on the owner
      \end{itemize}
    \item Both have roles in OSI2 design
  \end{itemize}
\end{frame}

\begin{frame}
  \frametitle{Use of Inheritance: Factories}
 
  \begin{itemize}
  \item User interacts with a pure virtual parent interface class
  \item A \emph{factory} class provides a \lstinline|createObject()|
    method that returns an appropriate child object
  \item Multiple implementations can be derived from the interface
  \item Child type can be decided at runtime
  \item Key design concept for plugin management
  \end{itemize}
\end{frame}

\begin{frame}
  \frametitle{Use of Composition: Bridges}

  \begin{itemize}
  \item Full set of primitive operations defined in owned class
  \item User interacts with a collection of high-level methods
  \item High-level methods implemented via calls to the owned
    object's primitives
  \item Different implementations of primitives can be instantiated
    using factories
  \item Different collections of high-level operations can share
    primitive implementations
  \item Concept used in ModelAPI and SolverAPI, which own solver
    shims to accomplish actions
  \end{itemize}
\end{frame}

\subsection{Proof of Concept: Incorporating OSI V1 Solver
  Interfaces}

\begin{frame}
  \frametitle{Concepts in Action: \lstinline|Osi1API|}

  \begin{itemize}
  \item Multiple inheritance
  \item Backward compatibility
  \item Ease of implementation when the user API matches the solver API
  \end{itemize}

\end{frame}

\begin{frame}[fragile]
  \frametitle{\lstinline|Osi1API| on the User Side}

  \begin{itemize}
  \item A new class that declares all \lstinline|OsiSolverInterface|
    methods as pure virtual methods
  \item
\begin{lstlisting}
class Osi1API : public API
{
  ...
  virtual void initialSolve() = 0;
  ...
}
\end{lstlisting}
  \end{itemize}
\end{frame}

\begin{frame}[fragile]
  \frametitle{\lstinline|Osi1API_ClpHeavy| on the Plugin Side}

  \begin{itemize}
  \item A new class that inherits all \lstinline|OsiSolverInterface|
    methods.

  \item
\begin{lstlisting}
class Osi1API_ClpHeavy
: public API, public OsiClpSolverInterface
{
  ...
  inline void initialSolve()
  { OsiClpSolverInterface::initialSolve(); }
  ...
}
\end{lstlisting}
  \end{itemize}
\end{frame}

\section{The Frontier: Open Design Decisions}

\begin{frame}
  \frametitle{Sample Use Case}

  Suppose we want to build and solve a model:

  \begin{itemize}
  \item Develop a model with a plugin specialized for model
    development.  \lstinline|ModelAPI| provides methods to create
    and maintain a model instance
  \item Solve the model with a plugin specialized for solving the
    model.  \lstinline|SolverAPI| provides methods to determine an
    optimal solution to a model instance
  \item Continue with cycles that modify and resolve the model.
    Need both APIs.
  \end{itemize}

  How should the objects implementing these APIs communicate?
\end{frame}

\begin{frame}
  \frametitle{Integrated Model}

  Suppose the solver plugin can support both the \lstinline|ModelAPI| and the
  \lstinline|SolverAPI|:
  \begin{itemize}
  \item OSI2 provides support for capability upgrades.
  \item Unload the model from the object implementing the
    \lstinline|ModelAPI| and load it into the object implementing
    the \lstinline|ModelAPI| and \lstinline|SolverAPI|.
    \begin{itemize}
    \item Explicitly invoke API load and unload operations.
      Implementation within OSI2.
    \item Use a 'copy constructor'.  Implemented by the solver shim.
    \end{itemize}
    \pause 
  \item Coordination between the model and the solver is handled by
    the solver object and the shim, because the solver is holding
    the only representation of the model.
  \item This is a good match to the current state of the art in
    solver implementation.
  \item The disadvantage is that combinations of APIs are limited to
    strict augmentation.
  \end{itemize}
\end{frame}

\begin{frame}
  \frametitle{Interacting Objects}

  Suppose that the solver does not support an integrated model; it expects to
  use an external model.
  \begin{itemize}
  \item OSI2 would provide support for synchronization between
    interacting objects. This is nontrivial, and there would be some
    overhead, but it's possible.
  \item If the solver were prepared to work with an external model,
    there would be no overhead beyond synchronization.
  \item In the current state of the art for solvers, there would be
    at least two copies of the model and multiple repetitions of
    copying the model from the \lstinline|ModelAPI| object to the
    \lstinline|SolverAPI| object.
  \item The advantage is that the interactions between objects are
    not constrained.
\end{itemize}
\end{frame}

\begin{frame}
  \frametitle{What Price Freedom?}

  The big questions:
  \begin{itemize}
  \item Will there be sufficient use cases to justify the costs of
    the interacting objects model?
  \item Will developers develop to this model if it's available?
  \end{itemize}
\end{frame}

\end{document}
