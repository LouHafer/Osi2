\documentclass{beamer}
\usepackage{amsmath,amssymb,listings}

\title{The COIN-OR \\ Open Solver Interface 2.0 \\ Redux \\[\jot] \tiny
  (Zombies Attack!)}
\author{Lou Hafer\inst{1} \and Matthew Saltzman\inst{2}}
\institute{
  \inst{1}
  Department of Computer Science \\
  Simon Fraser University
  \and
  \inst{2}
  Department of Mathematical Sciences \\
  Clemson University.
}
\date{INFORMS Annual Meeting \\ Austin, Texas \\ November 10, 2010}

\begin{document}
\lstset{language=C++}
\begin{frame}
  \titlepage
\end{frame}

\begin{frame}
  \frametitle{What is OSI?}

\begin{itemize}
\item A cross-solver API
\item Lower level than most solver APIs
  \begin{itemize}
  \item Instance management
  \item Algorithm control (e.g., pivot-level simplex) is a goal
    (honored more often than not in the breach)
  \item Intended as a ``crossbar switch'' to connect applications to
    solvers
  \end{itemize}
\item One of the original COIN-OR projects (a product of impetuous
  youth and inexperience)
\end{itemize}
\end{frame}

\begin{frame}
  \frametitle{A Fractured History}

  \begin{itemize}
  \item Original goal for OSI: ``solver independent''
    \begin{itemize}
    \item Too ambitious---can't reproduce outcomes reliably with
      different solvers
    \end{itemize}
    \pause
  \item More achievable: ``solver agnostic''
    \begin{itemize}
    \item Common way to interact with solvers, but don't enforce
      solver behavior
    \item Success (more or less) at source code level
    \end{itemize}

  \end{itemize}
\end{frame}

\begin{frame}
  \frametitle{Source Level Control}
  \begin{itemize}
  \item \texttt{OsiXxxSolverInterface} derived from
    \texttt{OsiSolverInterface} for various values of \texttt{Xxx}
    ($\text{\texttt{Xxx}} \in \{\text{\texttt{Clp}}, \text{\texttt{Cpx}},
  \text{\texttt{Grb}}, \dots \}$).
  \item User program instantiates concrete
    \texttt{OsiXxxSolverInterface} object
  \item So \texttt{Xxx} needs to be specified in source
  \item Selection can be controlled by compiler directives
    (\texttt{\#ifdef}) 
  \item Specified solver engine library required at link time
    \pause
  \item Solver engine library must ship with user's binary!
    \begin{itemize}
    \item For example, our CBC binary must ship with CLP as LP
      solver, because we can't ship other LP solvers.
    \item An end user who wants another solver engine must build
      from source
    \end{itemize}

  \end{itemize}
\end{frame}

\begin{frame}
  \frametitle{Shared Library for Solver Engine}

  \begin{itemize}
  \item If solver libraries are shared (\texttt{libxxx.so} on Unix,
    DLL on Windows)
    \begin{itemize}
    \item We can ship without solver libs---they are linked at load time
    \item Still need to specify solver in source code
    \item User still needs to have solver libs
    \item CPLEX isn't shipped as shared lib!
    \end{itemize}
  \end{itemize}
\end{frame}

\begin{frame}
  \frametitle{Dynamic Loading of Solver Engine}
  \begin{itemize}
  \item \texttt{dlopen()}, \texttt{dlsym()}, etc., in Linux, other
    calls in Windows and other Unix systems
  \item There are cross-platform libraries for this task (GNOME
    glib, GNU libtool)
  \item Solver Engine loaded at runtime, not needed at link time
  \item But still tied to solver in source
  \end{itemize}
\end{frame}

\begin{frame}
  \frametitle{Plugins: An Alternate Definition of ``Solver Independence''}

  \begin{itemize}
  \item Delay decision of what solver to instantiate until runtime
    \begin{itemize}
    \item User declares abstract interface object (\texttt{OsiSolverInterface})
    \item Asks factory object to create a specified concrete implementation
    \end{itemize}
  \item Solver engine loaded dynamically when implementation object
    is created
  \item Now we can ship CBC (say) and let the end user decide at runtime
    which LP engine to use
  \item Plugin builder and user need access to solver engine
    \pause
  \item Now we're breaking source code compatibility\dots
    \pause
  \item \dots so we can think about what we would do if we were
    starting with a clean slate
  \end{itemize}
\end{frame}

\begin{frame}
  \frametitle{What Else is Wrong?}

  \begin{itemize}
  \item Front and back ends can get out of sync
  \item Interface changes break everything
  \item Extensions are difficult
  \item Feature-complete new shims are painful to implement
  \item No upgrade path
  \item Too many tasks are implemented in the shim layer (e.g.,
    caching)---no way to implement common code
  \item No way for caller to know what capabilities are available or
    missing
  \item \dots
  \end{itemize}
\end{frame}

\begin{frame}
  \frametitle{What do we want?}

  \begin{itemize}
  \item Writing shims should be straightforward (not much harder than
    other APIs)
  \item Using the interface should be straightforward (not much harder
    than using an unwrapped solver)
  \item Performance penalty should be minimal
  \item The interface should provide a useful set of capabilities
  \item The interface should be extensible
    \begin{itemize}
    \item New capabilities should be easy to offer through the interface
    \item Hooking the solver directly should truly be a last resort
    \end{itemize}
  \end{itemize}
\end{frame}

\begin{frame}
  \frametitle{Decouple Interface from Implementation}

  \begin{itemize}
  \item In C++, this is a matter of programmer discipline
  \item Necessary to implement plugins
  \item Necessary to implement extensions
  \end{itemize}
  \pause
  \begin{itemize}
  \item Abstract base class exposes only public interface---defines
    module semantics
  \item Concrete implementation object derived from abstract base class
  \item User asks factory method to return concrete object
  \item All implementation details and private data are hidden
  \end{itemize}
\end{frame}

\begin{frame}
  \frametitle{Modularization}

  \begin{itemize}
  \item Clusters of related methods to accomplish tasks
    \begin{itemize}
    \item Core instance management
    \item Presolve
    \item Linear algebra, basis management
    \item Simplex
    \item B\&C control
    \item \dots
    \end{itemize}
  \item Capabilities managed via a map (name, version, factory)
    \begin{itemize}
    \item Name defines semantics via abstract base class
    \item Loading an interface returns a pointer to a concrete object
    \item Upgrades (for developer)
    \item Fallbacks (for user)
    \end{itemize}
  \item Modules associated with an instance need to match solver engine
  \end{itemize}
\end{frame}

\begin{frame}
  \frametitle{Inter-Module Communication}

  \begin{itemize}
  \item Modules are users too
    \begin{itemize}
    \item Need access to capabilities
    \end{itemize}
  \item Incoming module responsible for replacing existing module
    with the same functionality
    \begin{itemize}
    \item Extract data
    \item Replace capabilities
    \item Unload old module
    \end{itemize}

  \end{itemize}
\end{frame}

\begin{frame}
  \frametitle{Callbacks}

  \begin{itemize}
  \item C engine callback handled by registering a function with
    specific signature
  \item C++ engine callback is method derived from virtual base
    method
  \item Different engines define different categories of callbacks,
    provide different types of information, and allow different sets
    of actions
  \item OSI could implement a limited set of callback actions (check
    for abort flag and abort) in a common set of hooks
  \item Much more than that requires exposing solver-specific interfaces
  \end{itemize}
\end{frame}

\begin{frame}
  \frametitle{Parameter Management}

  \begin{itemize}
  \item Infinite variety
  \item Might be able to identify a set of common ones
  \item Map/table?
  \item Probably need to expose solver-specific interface for less
    common settings
  \end{itemize}
\end{frame}

\begin{frame}
  \frametitle{Other matters}

  \begin{itemize}
  \item Message handling
  \item Interactions with other COIN-OR components
  \item ???
  \end{itemize}
\end{frame}

\begin{frame}
  \frametitle{Lawyers, Guns, and Money}

  \begin{itemize}
  \item Plugins mitigate license compatibility issues
  \item GPL requires any program that ``includes'' GPL code must be GPL
  \item But plugins are not ``included'' in programs that use them
  \end{itemize}

\end{frame}

\begin{frame}
  \begin{center}
    Questions?  Suggestions?
  \end{center}
\end{frame}
\end{document}








\begin{frame}
  \frametitle{Outline}

  \begin{itemize}
  \item What's wrong with the current API?
  \item Dynamic Plugin Architecture
  \item Factories and Run-Time Instantiation
  \item Callbacks: Function-Pointer and Object-Based Approaches
  \item Extending the Basic Interface: OsiSimplex, etc.
  \end{itemize}
\end{frame}

\begin{frame}
  \frametitle{What's Wrong with the Current OSI?}

  The current OSI architecture is the original one from 2000.  It
  has proved to be useful in many contexts, but it has several
  properties that make it difficult to use and maintain.
  \begin{itemize}
  \item Back-end interfaces are created by inheritance from a base
    class (front end).
    \begin{itemize}
    \item Front-end and back-end interfaces can get out of sync.
    \item Too many tasks are implemented in the back end.
    \item Extending interfaces is difficult.
    \item Separation between instances and algorithms is problematic.
    \end{itemize}
  \item Solver interfaces are required at compile time.
    \begin{itemize}
    \item Control of which interfaces are needed is handled in
      source code, or by preprocessor directives (\texttt{\%ifdef}).
    \end{itemize}
  \item Solver libraries are required at link time.
  \end{itemize}
\end{frame}

\begin{frame}
  \frametitle{Plugin Architecture for Dynamic Solver Engine Linking}
  \begin{itemize}
  \item Decide which back-end interface and solver engine at runtime.
  \item Load just functions needed.
  \item No need to include engine library references with binary
    distributions.
  \item Windows DLLs and *nix static, shared, and dynamic libraries
    can be handled.
  \item C++ and C interfaces are possible.  With C++, all class
    information is exposed across the interface.  With C, objects
    can be wrapped to pass across.
  \end{itemize}
\end{frame}

\begin{frame}
  \frametitle{Callbacks and Event Handlers}

  Opportunity for the user to collect information about algorithm
  progress or take action at certain points in the algorithm or on
  the occurrence of certain external events.
  \begin{itemize}
  \item C solvers allow the user to register functions with the
    solver to be called at those points.
  \item C++ solvers call virtual functions at those points.  Users
    can derive replacement functions to achieve the desired result.
  \item A portable callback framework is a challenge.  Full callback
    flexibility probably requires exposing the solver engine's
    callback interface.
  \item Portability for callbacks using the intersection of common
    callback interfaces can be done.
  \item Hiding the C mechanism from C++ programs is possible.
    Hiding the C++ mechanism from C programs is possible if the set
    of actions is fixed (e.g., just abort).
  \end{itemize}
\end{frame}
\end{document}
